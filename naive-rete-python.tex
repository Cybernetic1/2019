\input{../YKY-preamble.tex}

\usepackage[CJKspace]{xeCJK}
\setCJKmainfont[BoldFont=SimHei,ItalicFont=KaiTi]{SimSun}
\usepackage{color}

\usepackage{mathtools}
\usepackage{hyperref}

\title{Comments to the Naive Rete python code}
\author{YKY, general.intelligence@gmail.com}

\begin{document}
\maketitle

\textbf{WME} = working memory element.

\textbf{Alpha} network: performs the ``decision tree'' tests.

\textbf{Alpha memory} = AM.  Store sets of WMEs, ie, \textit{instantiations}.

\textbf{Beta} network: checks consistency of \textbf{variable bindings} between conditions.

Beta memories store sets of \textbf{tokens}, each token representing a sequence of WMEs --- specifically, a sequence of $k$ WMEs, satisfying the first $k$ conditions (with consistent variable bindings) of some production.

\textbf{Production} node = \textbf{p-node} = firing of an action.

\textbf{ncc} = negated conjunctive conditions.  Tests for the absence of a certain combination of WMEs.

\textbf{Activation} means passing a new WME down the rete network.

\textbf{Left} activation = activation of some node from another node in the \textbf{beta} network

\textbf{Right} activation = activation of some node from the \textbf{alpha} network

\textbf{Tokens} = conjunctions of WMEs.  It's stored as a tree with each node holding 1 new WME and its parent.

Token has \textbf{join\_results} which is specifically for \textit{negative} nodes.  Negative nodes are just like ordinary join nodes, but instead of propagating the results of the join down the network, it stores them in the ``results'' memory in each token.  A token is thus propagated down the network if and only if its local result memory is \textit{empty}, since this indicates the absence of any such WME in working memory.

The \textbf{join\_results} store NegativeJoinResults.  Each NJR has a WME and its owner which is a token.

WMEs also have a field to store NJRs.  This is for handling WME \textit{removal}.

\section{Bug re: remove(x) but x not in list}

Seems that wme.negative\_join\_results does not contain enough NJRs for removal.  The fix is ``pop()''.

\section{Bug re ncc\_results: token.owner = None}

token.node = NCC Partner node

token.owner = None \hspace{2cm} \{ the token whose local memory this result resides \}

Solution: condition ``else'' not handled correctly in NCC partner left-activation.

\section{Bug: child.amem.successors.remove(child), child not in list}

Currently solution is to comment out this part of the code.  (For some unknown reason this part is not implemented by the NaiveRete author).

\section{Fixing the within-condition variable-binding problem}

Starts with:
\begin{equation}
{\footnotesize
\verb! self.alpha_root.activation(wme) !
}
\end{equation}
in \code{network.py}.

\end{document}
