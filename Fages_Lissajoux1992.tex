\input{../YKY-preamble}

\title{Operational semantics and compilation of production systems \\
--- Fages and Lissajoux's 1992 paper on the Rete algorithm}

\author{Francois Fages and R\'{e}mi Lissajoux \\
translated by YKY (Yan King Yin) \\ {\footnotesize general.intelligence@gmail.com}}

\begin{document}

	\setlength{\parindent}{0pt}
	\setlength{\parskip}{2.8ex plus0.8ex minus0.8ex}
	
	\maketitle
	
\begin{abstract}
We present an operational semantics, in the formalism of inferences rules a la
Plotkin, for production systems like OPS5 [For81]. We show how these programming
languages can be compiled from their operational semantics. We describe how sequential
and parallel versions of the Rete algorithm [For82] are formulated and their
implementations proved. We show that simple modifications of the formal system described
we obtain the Treat algorithm [Mir90] and other optimisations of Rete.
\end{abstract}

\textbf{Keywords:} production systems, compilation, operational semantics, rule based languages

\section{Introduction}

The development of expert systems has contributed to the emergence of the pure concept
declarative programming software, which can be summarized by Robert Kowalski:
\begin{equation}
\mbox{Algorithm = Logic + Control}
\nonumber
\end{equation}
Rule programming is the best example of this programming style.
Unlike a program written in a procedural or functional language in
which the sequence of instructions is explicit, a single rule base
the logical part of the processing, the sequencing of the instructions is defined
separately and is based on general techniques for finding a solution. The
tools generating expert systems are languages ​​(or overlanguages) of pro-
declarative grammar. The use of these tools makes it possible to develop effectively
applications and benefit from powerful compilation techniques. The fact that
these techniques ensure excellent performance on standard processors is
a decisive point for their integration into conventional computer systems,
The relative failure of dedicated machine projects or massively parallel machines
for the expert systems shows that the problem is of a purely
software.

In fact, rule-based languages now make it possible to reach
(in terms of execution time) allowing the use of these languages
in "real-time" applications (see [FL91]). These results are due as much to
The technological evolution of equipment that has improved the techniques of
lation. The performance report between the original version of Ops5 (developed
Carnegie Melion University, OpS5 is the ancestor of a number of rule-based languages
existing to date [For81)) and the latest version completed [GFK 88] is of two orders 
of magnitude (\& equal machine) [ML91].

To develop and master these compilation techniques more and more
complex, it is necessary to have a good understanding of these languages. That's what
allows to achieve a semantic study of language, be it semantics
which formalizes the execution of a program, or logical semantics
which characterizes in an abstract way the result of the execution.

Our effort tends to give a complete semantic analysis of the production systems
(of the OPS5 family), and more particularly of the XRete language! ([FL91, Tho91a,
Tho91b, Fag87] used for writing real-time embedded merge applications
data interpretation or interpretation (SF88) XRete combines the use of rules
of production and rules of deduction whose logical semantics are described by
elsewhere [FL91, Fag90b, Fag90a].

The formalization of operational semantics makes it possible to describe mathematically
running programs, defining and validating compilation algorithms
or study for example their parallelization, The purpose of this article is to give
a complete operational semantics of production systems and to derive from them
effective Timplémentation via the Rete algorithm used. Such an approach provides
a proof of the compiler that performs these transformations in practice. We show
also that the Treat algorithm proposed by D. Miranker as an alternative to
Rete [Mir90] can be defined using the same method by a simple modification
of the formal system presented, This method is sufficiently general to be used
both for parallel and sequential implementations. On the other hand, we
will see that the optimizations of Rete generally proposed (hash-code, permutation
single and multi-schema tests [GD84]) are easily defined in this formalism.

The plan is as follows: in a first part we briefly present the
languages \& rules of production, and give their operational semantics.
We then derive by successive refinements from this operational semantics
an algorithm (Rete [For82]) that implements it; we give proof
correction of this algorithm, in its sequential and parallel versions. We
let's end with a description of the different existing and possible implementations
of these languages.

\section{Operational semantics of production systems}

\subsection{Production systems}

A production system, denoted $\langle W, RB \rangle$, consists of a \textbf{rule-base} $RB$,
also called program, and a base of facts or objects $W$, also called \textbf{working memory}.

The ensemble of facts / objects will be identified in this article with the
\textbf{Herbrand base} $B_H$ (ie. all closed atomic propositions)
formed on an alphabet of symbols of constants, functions, and first-order predicates.

A production rule is a triple:
\begin{equation}
\pi \rightarrow \alpha, \rho
\nonumber
\end{equation}
where $\pi$ expresses a condition on the facts-base, $\alpha$ represents a series of facts \textbf{added} and
$\rho$ a series of facts \textbf{removed}.  More precisely $\pi, \alpha$ and $\rho$ can be identified with sequels
of literals of first-order logic, that is, atomic propositions
(called \textbf{schemas}), or the negation of atomic propositions, which may contain
variables. A production rule has the following form:
\begin{equation}
A_1, ..., A_k, \neg B_1, .., \neg B_l \rightarrow C_1, ... C_m, \neg D_1, ..., \neg D_n
\nonumber
\end{equation}
with the generally supplementary restriction that the variables of the schemas $B_j, 
C_j, D_j$ appear in the $A_i$'s.

"Negation by default" means to take $\neg A$ true if no instantiation of $A$ is
in the facts-base. An \textbf{instantiation} of a rule is a $k$-tuple of \textbf{working memory elements}, or WMEs,
$(o_1, ..., o_k)$ satisfying the left hand side of the rule,
that is to say, there exists a substitution $\sigma$ of the variables
of the rule verifying $A_i \sigma = o_i$, for $1 \le i \le k$,
and $\neg B_j \sigma$, for $1 \le j \le l$.  The negation on the right side of the rule
represents the action of deleting a fact ($\neg D_i \sigma$) from working memory.

The production rules allow expressing very general transformations
fact base while retaining an attractive declarative character. for example
the simplest sorting algorithm that can be imagined: iteratively
couples of badly ordered elements, written directly with a single rule of
production:
\begin{verbatim}
rule tri
?x : (elem indice=?i val=?v)
?y : (elem indice=?j &> ?i val=?w &< ?v)
->
modify (?x indice=?j)
modify (?y indice=?i)
endrule
\end{verbatim}
where variables are prefixed with ? (XRete syntax), the filtering in schemas is done
by the name of the attributes, and "modify" is equivalent to an addition of the fact followed by deletion.
When no more rules are triggerable, the stable state
obtained (the fixed point of the program) corresponds to a sorted configuration. The different
exchange sorting algorithms are obtained by specifying the triggering order
rules by a priority mechanism for example. It is remarkable that one
in this way obtain optimal complexity sorting algorithms on average
in $O(n  \log {n})$ where $n$ is the number of elements to be sorted, applying the techniques
described in this article (see section 4.2.1).

The various control strategies of the triggering order of the production rules
(the strategies \textit{lex, mea} [For81], static priorities, dynamics, meta-rules [Tho91a],
etc ...) are not covered in this article. We consider the production systems
as non-deterministic systems.

\subsection{Inference Cycle}

The content of the working memory represents the state of the system at a given moment
of computation, as the value of the variables represents the state of the execution of a program in a classical language. The rules are analogous to the ``program'' which
defines the changes made to the working memory. The formal definition of
inference cycle requires formally defining the filtering phase, i.e.
searching for rule instantiations.

A condition of the rule $\pi$ consists of a sequence of literals $L$. So we have
$\pi = \pi' . L$ or $\pi = true$.  We define the relation of \textbf{satisfaction} of a condition $\pi$ on
$W$ by a \textbf{partial instantiation} $(o_1, ..., o_k)$, denoted $(W, o_1, ..., o_k) \vDash \pi$, by the
following system:
\begin{eqnarray}
& W \vDash true \\
& \displaystyle \frac{(W, o_1, ..., o_{k-1}) \vDash \pi \quad o_k \in W \quad p(o_1, ... , o_k)}
{(W, o_1, ..., o_k) \vDash \pi . P} \\
& \displaystyle \frac{(W, o_1, ..., o_k) \vDash \pi \quad \neg \exists o \in W \quad p(o_1, ... , o_k, o)}
{(W, o_1, ..., o_k) \vDash \pi . P}
\end{eqnarray}
where $p(o_1, ..., o_j)$ denotes the conjunction of tests on $(o_1, ..., o_j)$ appearing in the
\textbf{schema} $P$.

A \textbf{instantiation} of rule of $\langle W, RB \rangle$ is a pair $(R, (o_1, ..., o_{ar(R)}))$ with
$R \in RB(o_1, ..., o_n) \in W^n$ such that $(W, o_1, ..., o_n) \vDash \pi(R)$.  Each instantiation of rule
defines a transition on the working memory.
\begin{equation}
W \rightarrow_{(R, (o_1, ... , o_n))} (W \ \rho(R, o_1, ..., o_n)) \cup \alpha(R, o_1, ..., o_n)
\nonumber
\end{equation}

The non-deterministic transition associated with the rule base $RB$, denoted by $\rightarrow$,
constitutes the \textbf{cycle of inference}. It is defined by:
\begin{equation}
\frac{R \in RB (W, o_1, ..., o_n) \vDash \pi(R) \quad W \rightarrow_{(R, (o_1, ..., o_n))} W'}
{RB \vdash W \rightarrow W'}
\end{equation}
Figure 1 schematically shows the inference cycle. [See original paper for figures]

Figure 1:

Filtering
\begin{equation}
\pi(R)(W, o_1, ..., o_n) = true
\nonumber
\end{equation}

Selection SEL
\begin{equation}
(R, (o_1, ..., o_n))
\nonumber
\end{equation}

Applying
\begin{equation}
W \rightarrow_{(R, (o_1, ..., o_n))} W'
\nonumber
\end{equation}

\subsection{Operational Semantics}

The operational semantics of a program $\langle W, RB \rangle$ is an \textbf{ensemble of stable states}
defined as the ensemble of terminal states for the relation $\rightarrow$, that is to say:
\begin{eqnarray}
|| \langle W, RB \rangle || = \{ W_s \in \mathcal{P} (B_H) | RB \vdash W \rightarrow^* W_s \nonumber \\
\mbox{and } \quad RB \vdash W_s \rightarrow W' \Rightarrow W_s = W' \} \nonumber
\end{eqnarray}
where the relation $\rightarrow^*$ is the transitive reflexive closure of the relation $\rightarrow$:
\begin{eqnarray}
& RB \vdash W \rightarrow^* W \nonumber \\
& \displaystyle \frac{RB \vdash W \rightarrow^* W' \quad RB \vdash W' \rightarrow W''}
{RB \vdash W \rightarrow^* W''} \nonumber
\end{eqnarray}

\section{Compilation}

\subsection{Rete algorithm: construction of the Rete graph}

The Rete [For82] algorithm is an efficient algorithm that calculates all
instantiations of rules incrementally. It was developed for OPS5. It is based on two
principles:

\begin{enumerate}
	
	\item Memorization: it is assumed that the number of modifications made
	to the working memory at each inference cycle is small compared to the
	size of the working memory. This assumption is verified in practice in
	many rule bases [GF83].  Therefore, all instantiations
	of rules in cycle $n$ is relatively little different from its state in cycle $n - 1$.
	So Rete's idea is to calculate only the modifications of the set of
	rule instantiations from one cycle to another. For this, the intermediate calculations (the
	partial rule instantiations) are stored.  
	
	\item Sharing: two rules can have some of their common conditions.
	The calculation relating to this sub-condition is factored between the two rules.
	
\end{enumerate}

Rete has been widely used for the efficient implementation of rule-based languages.

We derive from equations (2) and (3) a formal presentation of this algorithm.

First, the \textbf{predicate} $p(o_1, ..., o_j, o)$ associated with schema $P$ in equations (2) and (3)
can be \uline{broken down into two parts}:
\begin{equation}
p(o_1, ..., o_j, o) \equiv q(o) \wedge r(o_1, ..., o_j, o)
\nonumber
\end{equation}
where $q(o)$ denotes the conjunction of the \textbf{mono-schema tests} on $o$ appearing in
the schema $P$, and $r(o_1, ..., o_j, o)$ denotes the conjunction of \textbf{multi-schema tests} between
$(o_1, ..., o_j)$ and $o$ appearing in the schema $P_i$.  The decomposition of the test $p$ into $q$
and $r$ does not make any logical sense, but it permits to reduce the combination of
the search for $n$-tuples $(o_1, ..., o_j, o)$ that satisfy $p$ because we restrict the search for $o$
on objects verifying $q$.

This allows us to transform equations (2) and (3) into:
\begin{equation}
\frac{(W, o_1, ..., o_{k-1}) \vDash \pi \quad q(o_k) \quad r(o_1, ..., o_k) }
{(W, o_1, ..., o_k) \vDash \pi . P}
\end{equation}

\begin{equation}
\frac{(W, o_1, ..., o_{k}) \vDash \pi \quad \neg \exists o \in W \quad q(o) \wedge r(o_1, ..., o_k, o)}
{(W, o_1, ..., o_k) \vDash \pi . \neg P}
\end{equation}
The incremental character of the Rete algorithm comes from the storage of the $n$-tuples (respectively objects) that verifies the tests $\pi$ (respectively $q$) in the \textbf{left memory} $lm$ (respectively the \textbf{right memory} $rm$).

So, for a condition:
\begin{eqnarray}
\pi = P_1, ..., P_n \nonumber
\end{eqnarray}
the following left and right memories $lm$ and $rm$ are defined\footnote{$lm$ is called $\beta$-memories in the terminology of OPS5. $rm$ is called $\alpha$-memories (?) [YKY: second footnote is unreadable]}:

\textbf{Definition 1}

\begin{equation}
lm_i (E) = \{ (o_1, ..., o_{k_i}) \in E^{k_i} | ((o_1, ..., o_k), E ) \vDash \pi_i \} = E / \pi_i
\nonumber
\end{equation}

\begin{equation}
rm_i (E) = \{ o \in E | q_{i+1}(o) \} = E / q_{i+1}
\nonumber
\end{equation}

Equations (1), (5) and (6) naturally become:
\begin{equation}
\frac{(o_1, ..., o_{k-1}) \in lm_{i-1} (W) \quad o_k \in rm_{i-1} (W) \quad r_{i-1}(o_1, ..., o_k)}
{(o_1, ..., o_k) \in lm_i (W)}
\end{equation}
for a positive schema and $i \neq 1$, and:
\begin{equation}
\frac{(o_1, ..., o_{k}) \in lm_{i-1} (W) \quad \neg \exists o \in rm_{i-1} (W) \quad r_{i-1}(o_1, ..., o_k, o)}
{(o_1, ..., o_k) \in lm_i (W)}
\end{equation}
for a negative schema and $i \neq 1$, and:
\begin{equation}
lm_1 (W) = rm_0 (W)
\end{equation}

We now define the ensemble $rulem$ that contains the triggerable rule instantiations.

\begin{equation}
rulem(R, W) = \{ (R, (o_1, ..., o_n)) | (o_1, ..., o_n) \in lm_n(W) \}
\end{equation}

We can now write equation (4) as:
\begin{equation}
\frac{(o_1, ..., o_{k}) \in rulem (R) \quad W \rightarrow_{(R, (o_1, ..., o_k))} W' \quad R \in RB }
{W \rightarrow W'}
\end{equation}

The Rete algorithm uses a graph structure.

For each left and right \textbf{pile} $lm$ and $rm$ is associated a node of the graph identified by the pile to which it is attached.

On the other hand, we associate with each of the equations (7) and (8) a node called \textbf{juncture}.
The junctures corresponding to equation (7) are said to be positive; those corresponding to equation (8)
are said to be negative.  At each of these nodes are associated three \textbf{piles}: the \textbf{left} pile $lm_{i-1} (W)$, the \textbf{right} pile $rm_{i-1} (W)$ and the \textbf{resultant} pile $lm_i (W)$.

By connecting with an arc the nodes joining the nodes that the piles are associated with, we obtain a
graph.  By orienting the arcs of the junctures towards the resultant piles, and the left and right piles towards the junctures, we obtain an oriented graph (figure 2).

Figure 2. (Oriented rete graph)

Calculation of $\pi(o_1, ..., o_n)$:

\subsection{Rete algorithm: execution of the graph}

Formally, Rete associates an agent with each node of its graph. Schematically,
this agent corresponding to a juncture is responsible for maintaining the contents of the
resultant memory $lm_i$ depending on the evolution of the left and right memories $lm_{i-1}$ and $rm_{i-1}$.

These agents communicate by sending messages, called tokens, of the form
$\langle [+ -], e \rangle$. The element $e$ is either an object or an $n$-tuple. The sign + or - means
we want to add or remove the element. It corresponds to the two basic actions
defined in the right hand sides of rules.

The agents communicate with each other by synchronous communications. Arcs
of the graph are the communication channels of these messages. The communications
are in the direction of the orientation of the arcs.

We assume that there is a special Root agent responsible for receiving and distributing
messages received from the outside I In retracts an object $o$ from the working memory sends
[... unreadable ...]

We give the operational semantics of these agents.

For each agent, the rule is of the form
\begin{equation}
m? \stackrel{a}{\rightarrow} ms!
\end{equation}
where $m?$ is the message (token) received, $ms!$ the list of tokens sent in response and
$a$ the eventual action executed.

\subsubsection{Right memories $rm$}

\subsubsection{Left memories $lm$}

\subsubsection{Positive junctures}

\subsubsection{Negative junctures}

\subsubsection{node of rule}

\subsection{Sequential execution}

\subsection{Discrimination tree}

\subsection{Sharing}

\subsubsection{junctures}

\subsubsection{Discrimination}

\subsubsection{Genralization and example}

\subsection{The \textit{Treat} algorithm}

\section{Implementation}

\subsection{Sequential implementation}

\subsection{Optimizations}

\subsubsection{Sharing of local memories}

\subsubsection{Testing mono-schemas in junctures}

\subsection{Parallel implementation}

\subsection{juncture trees}

\subsection{Lazy rete}

\subsubsection{Other approaches to filtering}

\section{Conclusions}

\end{document}
